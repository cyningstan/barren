%
% Barren Planet - User Manual
% Main document source
%
% Copyright (C) Damian Gareth Walker 2022
% Manual Sections: Starting a New Game
%

% Chapter Heading
\chapter{Starting a New Game}

% Controls
\noindent
On loading {\it Barren Planet} for the first time, you will see the {\it Set up Game} screen. The window at the top left is where the game settings are shown. The window at the top right contains brief instructions. The small window at the bottom left is where a menu will appear when needed.

Before navigating your way around this screen you will need to know the game's controls. The game can be controlled entirely with the cursor movement keys and the {\it fire} key, which is your choice of: {\sc Ctrl}, {\sc Space} or {\sc Enter}.

On the {\it Set up Game} screen the up and down cursor controls move the highlight bar, and the left and right cursor controls change the highlighted option. The meaning of these options will be described shortly. Holding the {\it fire} key brings up a menu from which you can select an action using the up and down cursor controls; releasing the {\it fire} key will select the currently highlighted action from that menu.

The top row of the settings, {\it Game}, is only appropriate when there are saved games to load, and is described more fully in the {\it Game Management} chapter.

Beneath this is the {\it Campaign} setting. {\it Barren Planet} is supplied with a single campaign, called ``First Landing''. This campaign features the story of Nuvutech and Avuscorp and their initial fight to establish themselves on the planet Dapra. It consists of sixteen battles, although a single play-through will only involve some of these.

When other campaigns become available, you will be able to select them by changing the {\it Campaign} option. Future campaigns may feature different terrain and technology to play with, as well as new battles locations and possibly even new corporations or other warring factions. 

% Setting the Players
\section{Setting the Players}

\noindent
One or two players can take part in a game. This choice is made by choosing separately who will control Nuvutech and who will control Avuscorp. The choices are {\it Human}, {\it Computer} and {\it PBM}.

Human control allows the corporation to be played by a human player at this computer. If both sides are human players, then they take turns at the computer, one player finding something else to do while their opponent plays.

Computer control lets the computer take over control of that corporation. The computer will make its moves quickly, so there will not be much waiting for the human player between turns.

PBM stands for ``play-by-mail'', and allows two players at different computers to play against one another. The corporation marked {\it PBM} will be played by a player at another computer. Turns are passed between players on floppy disks or using electronic mail. Details of how to set up and conduct a play-by-mail game are given in the chapter on {\it Game Management}.

% Taking the First Turn
\section{On to Play}

\noindent
When the campaign and players have been selected, the game can be started by holding {\it fire} for the menu and selecting {\it Start Game}. The side who takes the first turn is randomly chosen.

If the first player happens to be controlled by the computer or a remote PBM player, then a screen show be shown telling the human player that the opponent is taking their turn. These are screens are covered in the {\it Ending the Turn} section of the {\it On the Battlefield} chapter.

If a human player moves first, then their briefing screen will be shown. If both players are human, this is preceeded by a Security Screen, which tells the players whose turn is about to begin. Selecting {\it Proceed} from the menu in this case will proceed to the Briefing Screen.
