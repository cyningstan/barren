%
% Barren Planet - User Manual
% Main document source
%
% Copyright (C) Damian Gareth Walker 2022
% Manual Sections: Reference
%
% Structure:
% Unit Types
% Terrain Types
%

% Chapter Heading
\chapter{Reference}

% Unit Types
\section{Unit Types}

% Introductory
\noindent
There are up to eight unit types in a {\bf \it Barren Planet} campaign. {\it First Landing} uses all eight, and they are as follows.

% Combat Droid
\subsection*{Combat Droid}

% Combat Droid Stats
\begin{center}
  \begin{tabular}{ c r l r l }
    \hline
    \multirow{4}{*}{\adjustimage{height=1cm,valign=m}{unit-combat-droid}}
    & Unit: & Combat Droid & Build cost: & 8 \\
    & Hit points: & 4 & Repair cost: & 4 \\
    & Attack power: & 2 & Attack range: & 1 \\
    & Armour: & 0 & Movement points: & 2 \\
    \hline
  \end{tabular}
\end{center}

% Combat Droid Description
\noindent
The {\it Combat Droid} is the backbone of your army. One of the cheapest units to produce, it can take a reasonable amount of damage, and its versatility allows it to take advantage of the protection provided by many types of terrain. This protection offsets it lack of armour. While the Combat Droid can traverse most types of terrain, it moves slowly.

% Hoverbug
\subsection*{Hoverbug}

% Hoverbug Stats
\begin{center}
  \begin{tabular}{ c r l r l }
    \hline
    \multirow{4}{*}{\adjustimage{height=1cm,valign=m}{unit-hoverbug}}
    & Unit: & Hoverbug & Build cost: & 8 \\
    & Hit points: & 3 & Repair cost: & 4 \\
    & Attack power: & 2 & Attack range: & 1 \\
    & Armour: & 1 & Movement points: & 3 \\
    \hline
  \end{tabular}
\end{center}

% Hoverbug Description
\noindent
The {\it Hoverbug} is a small anti-gravity fighting vehicle carrying a turret-mounted gun. Its ability to hover allows it to travel over most terrain with ease, although its top speed is not fast. Being a small vehicle, it cannot absorb much damage in battle, although it is lightly armoured.

% Air Fighter
\subsection*{Air Fighter}

% Air Fighter Stats
\begin{center}
  \begin{tabular}{ c r l r l }
    \hline
    \multirow{4}{*}{\adjustimage{height=1cm,valign=m}{unit-air-fighter}}
    & Unit: & Air Fighter & Build cost: & 8 \\
    & Hit points: & 2 & Repair cost: & 4 \\
    & Attack power: & 2 & Attack range: & 2 \\
    & Armour: & 0 & Movement points: & 6 \\
    \hline
  \end{tabular}
\end{center}

% Air Fighter Description
\noindent
The {\it Air Fighter} is a medium altitude flying unit suitable for attacking both air and ground units. As a flying unit it is the fastest unit available. The Air Fighter mounts a medium range laser allowing it to attack enemies at range. However, it is unarmoured and can take little damage, making it vulnerable if it allows enemy units to approach too closely.

% Ground Rover
\subsection*{Ground Rover}

% Ground Rover Stats
\begin{center}
  \begin{tabular}{ c r l r l }
    \hline
    \multirow{4}{*}{\adjustimage{height=1cm,valign=m}{unit-ground-rover}}
    & Unit: & Ground Rover & Build cost: & 16 \\
    & Hit points: & 4 & Repair cost: & 8 \\
    & Attack power: & 2 & Attack range: & 1 \\
    & Armour: & 2 & Movement points: & 4 \\
    \hline
  \end{tabular}
\end{center}

% Ground Rover Description
\noindent
The principal scouting vehicle is the {\it Ground Rover}. It is wheeled and can travel fast over open, flat terrain. It mounts a gun, and is heavily armoured, making it a good combat vehicle to take on other light units. Its combination of attack power and speed makes the Ground Rover an ideal choice for chasing down damaged retreating enemy units.

% Laser Tank
\subsection*{Laser Tank}

% Laser Tank Stats
\begin{center}
  \begin{tabular}{ c r l r l }
    \hline
    \multirow{4}{*}{\adjustimage{height=1cm,valign=m}{unit-laser-tank}}
    & Unit: & Laser Tank & Build cost: & 22 \\
    & Hit points: & 6 & Repair cost: & 11 \\
    & Attack power: & 4 & Attack range: & 2 \\
    & Armour: & 2 & Movement points: & 4 \\
    \hline
  \end{tabular}
\end{center}

% Laser Tank Description
\noindent
The {\it Laser Tank} is a heavy tracked combat unit. It mounts a turret bearing a powerful laser cannon, that can fire at medium range. Its bulk can take a lot of damage, and it is heavily armoured for extra protection. It can move rapidly across the battlefield.

% Laser Cannon
\subsection*{Laser Cannon}

% Laser Cannon Stats
\begin{center}
  \begin{tabular}{ c r l r l }
    \hline
    \multirow{4}{*}{\adjustimage{height=1cm,valign=m}{unit-laser-cannon}}
    & Unit: & Laser Cannon & Build cost: & 28 \\
    & Hit points: & 4 & Repair cost: & 14 \\
    & Attack power: & 6 & Attack range: & 3 \\
    & Armour: & 1 & Movement points: & 4 \\
    \hline
  \end{tabular}
\end{center}

% Laser Cannon Description
\noindent
The {\it Laser Cannon} is a mobile gun platform. It mounts a fixed high-power laser gun that can cause a lot of damage to units at medium range. It can take a moderate amount of damage and is lightly armoured, making it somewhat vulnerable to attack. This unit is best deployed in protective terrain where it can attack approaching enemies in relative safety.

% Gun Platform
\subsection*{Gun Platform}

% Gun Platform Stats
\begin{center}
  \begin{tabular}{ c r l r l }
    \hline
    \multirow{4}{*}{\adjustimage{height=1cm,valign=m}{unit-gun-platform}}
    & Unit: & Gun Platform & Build cost: & - \\
    & Hit points: & 3 & Repair cost: & - \\
    & Attack power: & 8 & Attack range: & 4 \\
    & Armour: & 0 & Movement points: & 1 \\
    \hline
  \end{tabular}
\end{center}

% Gun Platform Description
\noindent
A static unit, the {\it Gun Platform} is usually deployed in strategic positions to block access through narrow passes or to protect a nearby Mining Base. Its huge laser gun can target units at a very long range and causes an immense amount of damage; it can take out light units and occasionally even heavy units with a single hit. It is unarmoured and delicate, though, meaning that it is easily destroyed by any enemy that it allows to approach too closely. The Gun Platform is too sophisticated to be built or repaired on the battlefield.

% Mining Base
\subsection*{Mining Base}

% Mining Base Stats
\begin{center}
  \begin{tabular}{ c r l r l }
    \hline
    \multirow{4}{*}{\adjustimage{height=1cm,valign=m}{unit-mining-base}}
    & Unit: & Mining Base & Build cost: & - \\
    & Hit points: & 8 & Repair cost: & - \\
    & Attack power: & 2 & Attack range: & 2 \\
    & Armour: & 2 & Movement points: & 2 \\
    \hline
  \end{tabular}
\end{center}

% Mining Base Description
\noindent
The {\it Mining Base} is the most important unit in the game. It is a static unit, a fixed armoured building of significant bulk. This makes it difficult to destroy. It has very limited medium-range defensive power provided by a small laser gun mount. The armour and defensive power is insufficient to hold out against heavy combat units, so if the enemy threatens a Mining Base then it should be defended by other combat units.

% Mining Base Purpose
The use of the Mining Base is threefold. When built on a Crystal Node, it will generate 2 resource points per turn for its owner. With these resources it can repair damaged units that are adjacent. And finally, it can have the capability to use these resources to build new units on the battlefield. Resources gathered by multiple Mining Bases are pooled and can be spent at any Mining Base. Therefore Mining Bases deployed in places other than Crystal Nodes will not collect resources, but can still build and repair units if resources are stockpiled.

% Terrain Types
\section{Terrain Types}

% Introductory
\noindent
A campaign can use up to eight terrain types, and {\it First Landing} uses all eight of them. You will only see a handful of these in any one battle, as each map has its own character.

% Open Ground
\subsection*{Open Ground}

% Open Ground Stats
\begin{center}
  \begin{tabular}{ c r c c }
    \multicolumn{4}{c}{\bf Open Ground} \\
    & {\it Unit} & {\it Movement cost} & {\it Defence} \\
    \hline
    \multirow{8}{*}{\adjustimage{height=1cm,valign=m}{terrain-open-ground}}
    & Combat Droid & 1 & - \\
    & Hoverbug & 1 & - \\
    & Ground Rover & 1 & - \\
    & Air Fighter & 1 & - \\
    & Laser Tank & 1 & - \\
    & Laser Cannon & 1 & - \\
    & Gun Platform & - & - \\
    & Mining Base & - & - \\
  \end{tabular}
\end{center}

% Open Ground Description
\noindent
The majority of each map is taken up by Open Ground as, with the exception of air battles, this is the most practical type of terrain on which to deploy and move scouts and combat forces. Being open, it provides no protection to any kind of combat unit.

% Rocks
\subsection*{Rocks}

% Rocks Stats
\begin{center}
  \begin{tabular}{ c r c c }
    \multicolumn{4}{c}{\bf Rocks} \\
    & {\it Unit} & {\it Movement cost} & {\it Defence} \\
    \hline
    \multirow{8}{*}{\adjustimage{height=1cm,valign=m}{terrain-rocks}}
    & Combat Droid & 1 & 2 \\
    & Hoverbug & 1 & 1 \\
    & Ground Rover & 2 & 1 \\
    & Air Fighter & 1 & - \\
    & Laser Tank & 2 & 1 \\
    & Laser Cannon & 2 & 1 \\
    & Gun Platform & - & 1 \\
    & Mining Base & - & - \\
  \end{tabular}
\end{center}

% Rocks Description
\noindent
Some open ground on Dapra is strewn with large boulders. These can provide an obstacle for movement for larger units, but they can also provide protection from enemy fire. Combat Droids are especially adept at hiding behind the rocks.

% Ridge
\subsection*{Ridge}

% Ridge Stats
\begin{center}
  \begin{tabular}{ c r c c }
    \multicolumn{4}{c}{\bf Ridge} \\
    & {\it Unit} & {\it Movement cost} & {\it Defence} \\
    \hline
    \multirow{8}{*}{\adjustimage{height=1cm,valign=m}{terrain-ridge}}
    & Combat Droid & 2 & 2 \\
    & Hoverbug & 1 & 1 \\
    & Ground Rover & 2 & 1 \\
    & Air Fighter & 1 & - \\
    & Laser Tank & 2 & 1 \\
    & Laser Cannon & 4 & 1 \\
    & Gun Platform & - & 2 \\
    & Mining Base & - & - \\
  \end{tabular}
\end{center}

% Ridge Description
\noindent
The surface of Dapra is pockmarked by craters from falling meteorites. It also has many hills formed by tectonic activity. Both of these terrain types fall under the description of {\it Ridges}. Their steep approach is difficult to navigate for many units, only anti-grav and air units traversing them with ease. They offer varying amounts of protection to units that occupy them, being excellent places to deploy Combat Droids and Gun Platforms.

% Cliff Face
\subsection*{Cliff Face}

% Cliff Face Stats
\begin{center}
  \begin{tabular}{ c r c c }
    \multicolumn{4}{c}{\bf Cliff Face} \\
    & {\it Unit} & {\it Movement cost} & {\it Defence} \\
    \hline
    \multirow{8}{*}{\adjustimage{height=1cm,valign=m}{terrain-cliff-face}}
    & Combat Droid & - & - \\
    & Hoverbug & 2 & - \\
    & Ground Rover & - & - \\
    & Air Fighter & 1 & - \\
    & Laser Tank & - & - \\
    & Laser Cannon & - & - \\
    & Gun Platform & - & - \\
    & Mining Base & - & - \\
  \end{tabular}
\end{center}

% Cliff Face Description
\noindent
The same tectonic forces that created Ridges can also rupture the planet surface enough to form a {\it Cliff Face}, a rise so steep and high that only air and anti-grav units can navigate it. Units flying over the Cliff Face have no protection, and are therefore vulnerable. But sometimes they provide a route for these units to outflank the enemy.

% Magma Flow
\subsection*{Magma Flow}

% Magma Flow Stats
\begin{center}
  \begin{tabular}{ c r c c }
    \multicolumn{4}{c}{\bf Magma Flow} \\
    & {\it Unit} & {\it Movement cost} & {\it Defence} \\
    \hline
    \multirow{8}{*}{\adjustimage{height=1cm,valign=m}{terrain-magma-flow}}
    & Combat Droid & - & - \\
    & Hoverbug & 1 & - \\
    & Ground Rover & - & - \\
    & Air Fighter & 1 & - \\
    & Laser Tank & - & - \\
    & Laser Cannon & - & - \\
    & Gun Platform & - & - \\
    & Mining Base & - & - \\
  \end{tabular}
\end{center}

% Magma Flow Description
Dapra has very little surface water, most if not all of which is locked up in ice caps. Instead its ``rivers'' are formed of flowing magma. These {\it Magma Flows} ooze out from volcanic fissures, and are impassable to all units except anti-grav and air units. Like the Cliff Face, a Magma Flow can provide a route for these units to outflank the enemy.

% Volcano
\subsection*{Volcano}

% Volcano Stats
\begin{center}
  \begin{tabular}{ c r c c }
    \multicolumn{4}{c}{\bf Volcano} \\
    & {\it Unit} & {\it Movement cost} & {\it Defence} \\
    \hline
    \multirow{8}{*}{\adjustimage{height=1cm,valign=m}{terrain-volcano}}
    & Combat Droid & - & - \\
    & Hoverbug & - & - \\
    & Ground Rover & - & - \\
    & Air Fighter & - & - \\
    & Laser Tank & - & - \\
    & Laser Cannon & - & - \\
    & Gun Platform & - & - \\
    & Mining Base & - & - \\
  \end{tabular}
\end{center}

% Volcano Description
\noindent
The most dramatic manifestations of volcanic activity on Dapra are the {\it Volcanoes} themselves. These are active, spewing out fountains of lava and ash into the air. Not even air units can safely pass over them, so they form an impenetrable barrier to all units best admired from afar.

% Crystal Node
\subsection*{Crystal Node}

% Crystal Node Stats
\begin{center}
  \begin{tabular}{ c r c c }
    \multicolumn{4}{c}{\bf Crystal Node} \\
    & {\it Unit} & {\it Movement cost} & {\it Defence} \\
    \hline
    \multirow{8}{*}{\adjustimage{height=1cm,valign=m}{terrain-crystal-node}}
    & Combat Droid & 1 & 1 \\
    & Hoverbug & 1 & - \\
    & Ground Rover & 2 & - \\
    & Air Fighter & 1 & - \\
    & Laser Tank & 4 & - \\
    & Laser Cannon & 4 & - \\
    & Gun Platform & - & - \\
    & Mining Base & - & - \\
  \end{tabular}
\end{center}

% Crystal Node Description
\noindent
The {\it Crystal Nodes} are those places where the lucrative resources of Dapra pierce the planet surface, and become visible to surface and air units. The shard-like structures can only be exploited using the sophisticated equipment present in a Mining Base, but are large enough to provide moderate protection for Combat Droids that can hide in them. Like Rocks, the Crystal Nodes can be easily traversed by Combat Droids and anti-grav and air units, but are difficult of access for heavier units. Laser Tanks and Laser Cannon must move especially slowly not to have their tracks damaged by the hard and sharp features poking up from the ground.

% Alien Ruin
\subsection*{Alien Ruin}

% Alien Ruin Stats
\begin{center}
  \begin{tabular}{ c r c c }
    \multicolumn{4}{c}{\bf Alien Ruin} \\
    & {\it Unit} & {\it Movement cost} & {\it Defence} \\
    \hline
    \multirow{8}{*}{\adjustimage{height=1cm,valign=m}{terrain-alien-ruin}}
    & Combat Droid & 1 & 3 \\
    & Hoverbug & 2 & 1 \\
    & Ground Rover & 4 & 1 \\
    & Air Fighter & 1 & - \\
    & Laser Tank & - & - \\
    & Laser Cannon & - & - \\
    & Gun Platform & - & - \\
    & Mining Base & - & 1 \\
  \end{tabular}
\end{center}

% Alien Ruin Description
\noindent
Humans are not the first species to roam the surface of Dapra. During the course of their exploration, Nuvutech and Avuscorp come across structures which are obviously not natural. The nature of these structures has not yet been studied, and nothing is known about the alien race that built them. Were they once occupants of a more hospitable Dapra? Or were they visitors like the present ones, here to avail themselves of the planet's mineral riches?

% Current use
You won't get to find out any time soon. Because for the present, these structures are being used as protected terrain in the fighting between the two corporations. Perhaps the ultimate victor will invite researchers down to the planet, once the strife is over, to study them. Or what's left of them after these present battles have taken their toll.
